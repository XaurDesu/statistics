%%%%%%%%%%%%%%%%%%%%%%%%%%%%%%%%%%%%%%%%%
%  A bright and image filled report style, currently set up here for use with ILM report 8600-219.
%  Contains all that is required, glossaries, content management, references and good looks.
%
% The original template (the Legrand Orange Book Template) can be found here --> http://www.latextemplates.com/template/the-legrand-orange-book
% Original author of the Legrand Orange Book Template:
% Mathias Legrand (legrand.mathias@gmail.com) 
%
% Modifications made for ILM specific reporting
% 
%
% License:
% CC BY-NC-SA 3.0 (http://creativecommons.org/licenses/by-nc-sa/3.0/)
%%%%%%%%%%%%%%%%%%%%%%%%%%%%%%%%%%%%%%%%%
 
%%%%%%%%%%%%%%%%%%%%%%%%%%%%%%%%%%%%%%%%%
% How to use this
%
% Upload a file called FrontCover.jpg to become your new front cover - into the Pictures folder
% Upload files called Heading1.jpg, Heading2.jpg etc to become your new chapter headers - into the Pictures folder
% Make sure these images are the right size to fit their locations and use good quality images
%
% Locate the variables below and set your name, title etc.
%
% If you want to change text colour on the front cover, the areas required are commented below
% If you want to modify text and border colours for your chapter headers go into the structure.tex file and replace the name of the colour (set to ) with a new colour name (find and replace ctrl+f will do this for you).
%
% Add all references into references.bib
% Cite these references by using \cite{referenceName}
%
% Commonly used acronyms or industry specific terms should be added to the glossary
% These terms may then be referenced in the text using \gls{termName}
%
% Finally put some answers in there!
%
% 
% Note: This template is set up specifically for ILM reports, it can be modified for other forms of reports
%
%%%%%%%%%%%%%%%%%%%%%%%%%%%%%%%%%%%%%%%%
 
 
%----------------------------------------------------------------------------------------
%	SET THESE VARIABLES!
%----------------------------------------------------------------------------------------

\def\mytitle{Probability and Statistics} % Title of the ILM project

\def\author{Jaime Andres Torres B.} % Your name.. 
\def\id{github.com/XaurDesu} % Your unique identifier

\def\date{\today } % Today's date 


 
%----------------------------------------------------------------------------------------
%	PACKAGES AND OTHER DOCUMENT CONFIGURATIONS
%----------------------------------------------------------------------------------------

\documentclass[11pt,fleqn]{book} % Default font size and left-justified equations

\usepackage[dvipsnames]{xcolor}

\input{structure} % Insert the commands.tex file which contains the majority of the structure behind the template

\makeglossaries

%--------------------------------------------------------------------------

% Glossary entries

%--------------------------------------------------------------------------
\newglossaryentry{ETN}
{
    name = {Example Term Name (ETN)},
    description = {What does this term mean? Any examples of it? Further reading? References?}
}

%--------------------------------------------------------------------------

% Document begins here

%--------------------------------------------------------------------------

\begin{document}
\renewcommand{\bibname}{References} % Adds in the link to your references


%----------------------------------------------------------------------------------------
%	TITLE PAGE
%----------------------------------------------------------------------------------------

\begingroup
\thispagestyle{empty}
\AddToShipoutPicture*{\put(0,0){\includegraphics{Pictures/FrontCover.jpg}}} % Image background
\centering
\vspace*{11.3cm}
\par\normalfont\fontsize{35}{35}\sffamily\selectfont

\begin{center}
    % List of Latex Colour names here: https://www.overleaf.com/learn/latex/Using_colours_in_LaTeX
    \textbf{\color{Apricot} \mytitle}  % Modify the name of the colour used to suit your image
    
    \color{White}In a friendly manner % Modify the name of the colour used to suit your image
    
    \vspace*{0.5cm}
    \color{White}\author % Modify the name of the colour used to suit your image
    
    (\id)\par  
\end{center}

\endgroup

%----------------------------------------------------------------------------------------
%	COPYRIGHT PAGE
%----------------------------------------------------------------------------------------


\newpage
~\vfill
\thispagestyle{empty}

\noindent \textbf{Why take notes like this?}
\vspace{0.5cm}

\noindent I dunno, it's cool to do so i guess.

\vspace{1cm}

\noindent \textit{First release, \date} % Printing/edition date

%----------------------------------------------------------------------------------------
%	TABLE OF CONTENTS
%----------------------------------------------------------------------------------------

\chapterimage{Heading1.jpg} % Table of contents heading image

\pagestyle{empty} % No headers

\tableofcontents % Print the table of contents itself

%\listoftables %uncomment this if you want to print the list of tables at the start

\pagestyle{fancy} % Print headers again


%----------------------------------------------------------------------------------------
%	Glossary
%----------------------------------------------------------------------------------------

\chapterimage{Heading1.jpg} % Table of contents heading image

\printglossaries



%----------------------------------------------------------------------------------------
%	First set of related questions
%----------------------------------------------------------------------------------------
%--------------------------------------------------------------
%	More sections?
%----------------------------------------------------------------------------

\chapterimage{Heading2.jpg}
\chapter{Fundamentals.}

\section{Experiments and Events}

On the field of probability, we'll define an experiment as the process that generates a set
of data. But also more broadly, it is a process that generates any data that we might find
an observation or result with. We denote an experiment with the letter $ \zeta $, examples
of experiments include:

\begin{itemize}
    \item $\zeta$ 1: The number of rolls of a dice before landing on the number 6
    \item $\zeta$ 2: The number of calls from Bogota to New York being registered on the next hour, starting instantly.
    \item $\zeta$ 3: The result of a hipotecary credit (approved or rejected)
    \item $\zeta$ 4: The time between now and the next earthquake measured with 5.0 or more on Richter's scale somewhere in south america.
\end{itemize}

As we can see, a random experiment can cover a very broad category of measurements when we really think about it, however,
they all show three particular characteristics, they must be present at all times for a random experiment to be able to 
have itself called as random:

\begin{itemize}
    \item \textbf{Randomness: }We must not know the result of the experiment before it concludes
    \item \textbf{Unique results: }Every experiment must produce an unique result per instance.
    \item \textbf{Determinable results: }The result of an experiment must be able to be noticed and determined or otherwise categorized.
\end{itemize}

\subsection{Sample space}

A sample space, denoted as $ \Omega  $ or 'S' in most literature, is a set of possible results in a random experiment 
$ \zeta $. Such an experiment $ \zeta $ can have multiple sample spaces $ \Omega $. Such a space can be both finite and infinite,
in case of it being infinite, we must be able to distinguish wether or not it is numerable, in case it is we will
call such a space 'discrete' otherwise, we call it 'continuous'.

For an example on how this could be seen, let's take into consideration the following example:

$ \zeta $: The number of times a dice is rolled with a number greater than 3 that is odd

We can find two distinct sample spaces in this example, namedly:
$$
\begin{cases}
    \Omega_1 \text{:} \{ 1,2,3,4,5,6 \} \\
    \Omega_2 \text{:} \{ \text{even, odd} \}    
\end{cases}
$$

both of these spaces are described by ennumerating, however, we can also describe sample spaces via rules and mathematical notation,
for example, we could describe our first sample space in the former example as:

$$ \Omega_1: \{ x | 1 \le x \le 6 \} $$

A sample space is not necessarily tied to a data variable, and a data variable might be implied more than once on a dataset,
take this thought experiment, for example:

\textit{Ex. 1}

\paragraph*{In a probability and statistics class of 56 students, one 
student is selected as a representative to the Student Council and 
another as an alternate. Thus, the result of the experiment can be 
represented by the following tuple:}

\begin{center}
    (president, alternate)    
\end{center}

What's the size of the sample space here presented?

\paragraph{Solution:}

Even though the number of students is just 56, this doesn't mean the sample size is also
56, this is because we have a combination where we use the same dataset twice on two 
different random experiments, noted as:
$$
\begin{cases}
    \zeta_1 \text{: The election of a president.}\\
    \zeta_2 \text{: The election of an alternate.}
\end{cases}
$$

Both of them will have similar, but ultimately different sizes and the final result will be a combination
between the two. we could think of it as the total number of students, multiplied by the total number of
students minus one (since we have to select a president and the president themselves cannot be the alternate), 
therefore:

\begin{gather}
    len(S) = 56 * (56-1) \\
    len(S) = 56 * 55 \\
    len(S) = 3080
\end{gather}

the total number of possible results is 3080


\section{Definitions.}
\begin{enumerate}
    \item \textbf{Union}
    
    Between a set $ A $ and a set $ B $, we can say an Union is the collection of items that can be found in
    either set. it's marked as $ A \bigcup  B $


    \item \textbf{Intersection}

    Between a set $ A $ and a set $ B $, we can say an intersection is the collection of items that can be found in
    both sets. it's marked as $ A \bigcap  B $

    \item \textbf{Compliment}
    
    Noted as $ A^c $ or $ A' $ on a set '$ A $', a compliment is in essence the inverse 
    of said set. Every item not in $ A $ will be part of its compliment.
    
    \item \textbf{Event}
    
    An event is defined as the result of a random experiment. Once the experiment $ \zeta $
    is finished we denote the result as a subset 'B' of the sample space defined for $ \zeta $

    \item \textbf{Impossible Event}
    
    An event with no chance of ocurring with the parameters given. Such an event should be
    noted as $P() = \varnothing = 0 $
\end{enumerate}

\chapterimage{Heading2.jpg}
\chapter{Counting Methods.}

We can count the elements in a set with different techniques, depending on what our objective is with this 


% Simply upload additional images Heading5.jpg, Heading6.jpg etc. into the pictures folder

% \chapterimage{Heading5.jpg}
% \chapter{Third Set of Questions}


%----------------------------------------------------------------------------------------
%	References
%----------------------------------------------------------------------------------------

\chapterimage{Heading4.jpg} % Chapter heading image

\bibliographystyle{plain} % Change this to IEEE or Harvard etc.
\bibliography{references}


\end{document}